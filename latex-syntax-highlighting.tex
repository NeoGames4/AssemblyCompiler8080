% This package is enables more color settings
\usepackage[dvipsnames]{xcolor}

% Import the listings package to display code
\usepackage{listings}

% Label code sections with "Code" (e. g. "Code 1: […]", "Code 2: […]", ….)
\renewcommand{\lstlistingname}{Code}
% \lstlistoflistings title -> "Codeverzeichnis"
\renewcommand{\lstlistlistingname}{\lstlistingname verzeichnis}

% Syntax highlighting, define assembly
\lstdefinelanguage{ASM}{
    % ELEMENTS
    morekeywords    = [1]{MVI,MOV,INR,DCR,CMP,ADI,DAD,ANA,ORA,XRA,JMP,JZ,JNZ,IN,OUT,HLT}, % instructions
    morekeywords    = [2]{A,B,C,D,E,H,L,X,R,DAC,ADC}, % registers
    sensitive       = false, % case-sensitivity
    morecomment     = [l]{;}, % l is for line comment
    morestring      = [b]", % defines that strings are enclosed in double quotes
    % STYLE
    frame           = tb, % top bottom line
    captionpos      = b, % captions at the bottom
    numbers         = left, % line numbers on the left
    tabsize         = 4, % one tab has the width of four spaces
    columns         = fixed, % monospaced
    extendedchars   = true, % 256 instead of 128 characters
    basicstyle      = \ttfamily\small,
    breaklines      = true,
    literate        = {ö}{{\"o}}1 % display ä, ö, ü
                      {ä}{{\"a}}1
                      {ü}{{\"u}}1,
    % SYNTAX HIGHLIGHTING
    keywordstyle    = [1]\color{blue},
    keywordstyle    = [2]\color{black},
    commentstyle    = \color{CadetBlue},
    numberstyle     = \color{white!50!black},
    identifierstyle = \color{green!50!black}
}

% Example code
\begin{lstlisting}[language=ASM, caption={Assembly source code for generating a sawtooth signal.}, label={lst:sawtooth}]
MVI A, 0  ; 3E \ 00 ; Set A to 0
F1:     ;     ;
INR A   ; 3C    ; Increase A by 1
OUT DAC   ; D3 \ 04 ; Output A at the DAC
JMP F1    ; C3 \ 02\00; Repeat from F1
\end{lstlisting}

% Put this at the end of your document to list all code snippets
\lstlistoflistings
